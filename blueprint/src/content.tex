% In this file you should put the actual content of the blueprint.
% It will be used both by the web and the print version.
% It should *not* include the \begin{document}
%
% If you want to split the blueprint content into several files then
% the current file can be a simple sequence of \input. Otherwise It
% can start with a \section or \chapter for instance.

\begin{definition}[完全数]\label{perfect}\lean{Perfect_}\leanok~\

自然数 \(n\) が完全数であるとは, \(n\) の真の約数(\(n\) 自身を除く正の約数)の総和が \(n\) に等しいとき,すなわち
\[
\sum_{\substack{d \mid n\\ d < n}} d = n
\]
\;\;\;が成り立つことをいう.

\end{definition}


\begin{definition}[約数関数]\label{sigma_div}\lean{sigma_div}\leanok~\

自然数 \(n\) について, \(n\) のすべての約数 \(d\) に対して \(d^k\) (\(k\) は自然数) を足し合わせた関数
\[
\sigma_k(n) := \sum_{d \mid n} d^k
\]
\;\;\;を \(k\) 次の約数関数という.特に, \(\sigma_1(n)\) は \(n\) の約数の総和を表す.

\end{definition}

\vspace{0.5\baselineskip}

今回は \(k = 1\) のときのみ考え, \(\sigma_1(n) = \sigma(n)\) と表す.

\vspace{0.5\baselineskip}

\(\sigma(n)\) は以下の性質をもつ.

\begin{lemma}\label{one_iff_sum_divisors_eq_one}\lean{one_iff_sum_divisors_eq_one}\leanok

\(n = 1\) のときに限り, \(\sigma(n) = 1\).

\end{lemma}

\begin{proof}
\(n = 1\) のとき, \(n\) の約数は \{1\}のみ. \(\therefore ~\sigma(n) = 1\).

逆に, \(\sigma(n) = 1\) のとき, \(n \neq 1\) とすると, \(\sigma(n) > 1\) となるため, \(n = 1\).

\end{proof}

\begin{lemma}\label{perfect_iff_sum_divisors_eq_two_mul}
\lean{perfect_iff_sum_divisors_eq_two_mul}\leanok\uses{perfect}

\(n\) が完全数のときに限り, \(\sigma(n) = 2n\).

\end{lemma}

\begin{proof}
\(s(n) = \sum_{\substack{d \mid n\\ d < n}} d = n\) とする.
定義2より, \(n\) が完全数のとき, \(s(n) = n\).

\(\sigma(n) = s(n) + n\) であるから, \(\sigma(n) = 2n\).

逆に, \(\sigma(n) = s(n) + n = 2n\) より, \(s(n) = n\).
よって, \(n\) は完全数.
\end{proof}

\begin{lemma}\label{prime_iff_sum_divisors_eq_succ}\lean{prime_iff_sum_divisors_eq_succ}\leanok

\(n\) が素数のときに限り, \(\sigma(n) =  1 + n\).

\end{lemma}

\begin{proof}
\(n\) が素数のとき, \(n\) の約数は \{1, n\} より, \(\sigma(n) = 1 + n\).

逆に, \(\sigma(n) = s(n) + n = 1 + n\) より, \(s(n) = 1\).
このとき, \(n\) は素数.
\end{proof}

\begin{lemma}\label{isMultiplicative}\lean{isMultiplicative_sigma}\leanok
乗法的関数である.すなわち, 互いに素な自然数 \(m, n\) に対して,

\(\sigma(mn) = \sigma(m) \cdot \sigma(n)\) が成り立つ.

\end{lemma}

\begin{proof}
\(m, n\) を互いに素な自然数とする.素因数分解の一意性から,
\[
m = \prod_{i=1}^{r}p_i^{a_i}, n = \prod_{i=1}^{r'}p_{r+i}^{a_{r+i}}
\]と一意に表せる.
\(m, n\) は互いに素より,任意の異なる \(i,j\in\{1, 2, \cdots, r, r+1, \cdots, r'\}\) について \(p_i \neq p_j\).

よって, \[mn = \prod_{i=1}^{r+r'}p_i^{a_i}\] が成り立つ.
ここで \(\sigma(m), \sigma(n)\) は,
\begin{align*}
\sigma(m) &= \sigma(\prod_{i=1}^{r}p_i^{a_i})\\
          &= (1 + p_1 + \cdots + p_1^{a_1}) \cdots (1 + p_r + \cdots + p_r^{a_r})\\
          &= \prod_{i=1}^{r}\sum_{b=0}^{a_i}p_i^{b}
\end{align*}

\begin{align*}
\;\;\;\;\;\;\;\;\;\;\;\;\;\;\;\sigma(n) &= \sigma(\prod_{i=1}^{r'}p_{r+i}^{a_{r+i}})\\
          &= (1 + p_{r+1} + \cdots + p_{r+1}^{a_{r+1}}) \cdots (1 + p_{r+r'} + \cdots + p_{r+r'}^{a_{r+r'}})\\
          &= \prod_{i=r+1}^{r+r'}\sum_{b=0}^{a_i}p_i^{b}
\end{align*}
となる.また \(\sigma(mn)\) は,
\begin{align*}
\sigma(mn) &= \sigma(\prod_{i=1}^{r+r'}p_i^{a_i})\\
           &= (1 + p_1 + \cdots + p_1^{a_1}) \cdots (1 + p_r + \cdots + p_r^{a_r})
              \cdot (1 + p_{r+1} + \cdots + p_{r+1}^{a_{r+1}}) \cdots (1 + p_{r+r'} + \cdots + p_{r+r'}^{a_{r+r'}})\\
           &= (\prod_{i=1}^{r}\sum_{b=0}^{a_i}p_i^{b}) \cdot (\prod_{i=r+1}^{r+r'}\sum_{b=0}^{a_i}p_i^{b})\\
           &= \sigma(m) \cdot \sigma(n)
\end{align*}

したがって, \(\sigma(mn)\) = \(\sigma(m)\sigma(n)\) が成り立つため, \(\sigma\) は乗法的関数である.
\end{proof}


\begin{definition}[メルセンヌ数]\label{Mersenne}\lean{Mersenne}\leanok~\

自然数 \(k\) に対して, \(M_k := 2^k - 1\) をメルセンヌ数と呼ぶ. \(M_k\) が素数であるとき,メルセンヌ素数という.

\end{definition}


\begin{lemma}\label{sigma_two_pow_eq_mersenne_succ}\lean{sigma_two_pow_eq_mersenne_succ}\leanok
\uses{Mersenne}
\(\sigma(2^k) = M_{k+1}\)

\end{lemma}

\begin{proof}
\begin{align*}
  \sigma_1(2^k) = M_{k+1} &\Leftrightarrow \sum_{d \mid 2^k} d = M_{k+1}\\
                          &\Leftrightarrow \sum_{d \mid 2^k} d = 2^{k+1} - 1\\
                          &\Leftrightarrow \sum_{d \mid (1+1)^k} d = (1+1)^{k+1} - 1\\
                          &\Leftrightarrow \sum_{d \mid (1+1)^k} d = (\sum_{i=0}^{k}(1+1)^i) \cdot 1 + 1 - 1
\end{align*}
\end{proof}


\begin{theorem}[Euclid I]\label{perfect_two_pow_mul_mersenne_of_prime}
\lean{perfect_two_pow_mul_mersenne_of_prime}\leanok
\uses{sigma_two_pow_eq_mersenne_succ, Mersenne, prime_iff_sum_divisors_eq_succ, isMultiplicative}~\

\(M_{k+1} = 2^{k+1} - 1\) が素数ならば, \(2^k \cdot M_{k+1}\) は完全数.

\end{theorem}

\begin{proof}
\begin{align*}
2^kM_{k+1} : Perfect &\Leftrightarrow \sum_{d \mid 2^kM_{k+1}} d = 2(2^kM_{k+1}) \wedge 2^kM_{k+1} > 0\\
                     &\Leftrightarrow \sum_{d \mid 2^kM_{k+1}} d = 2^{k+1}M_{k+1} \wedge 2^kM_{k+1} > 0\\
                     &\Leftrightarrow \sigma_1(2^kM_{k+1}) = 2^{k+1}M_{k+1} \wedge 2^kM_{k+1} > 0\\
                     &\Leftrightarrow \sigma_1(M_{k+1}) \cdot \sigma_1(2^k) = 2^{k+1}M_{k+1} \wedge 2^kM_{k+1} > 0~~~(\because~lemma2.1)\\
                     &\Leftrightarrow \sigma_1(M_{k+1}) \cdot M_{k+1} = 2^{k+1}M_{k+1} \wedge 2^kM_{k+1} > 0~~~(\because~lemma2.2)\\
                     &\Leftrightarrow \sum_{d \mid M_{k+1}} d \cdot M_{k+1} = 2^{k+1}M_{k+1} \wedge 2^kM_{k+1} > 0\\
&\text{\(M_{k+1}\) は素数より, \(\sigma_1(M_{k+1}) = 1 + M_{k+1} = 2^{k+1}\).}
\end{align*}
\end{proof}


\begin{lemma}\label{ne_zero_of_prime_mersenne}\lean{ne_zero_of_prime_mersenne}\leanok
\uses{Mersenne}
\(M_{k+1} = 2^{k+1} - 1\) が素数ならば, \(k \neq 0\).

\end{lemma}

\begin{proof}
背理法を用いる. \(k = 0\) と仮定する.

\(M_{k+1} : Prime = M_{0+1} = 1\) となるが, \(1\) は素数でない.よって矛盾.
\end{proof}


\begin{theorem}[Euclid I\hspace{-1.2pt}I]\label{even_two_pow_mul_mersenne_of_prime}
\lean{even_two_pow_mul_mersenne_of_prime}\leanok\uses{ne_zero_of_prime_mersenne, Mersenne}~\

\(M_{k+1} = 2^{k+1} - 1\) が素数ならば, \(2^k \cdot M_{k+1}\) は偶数.

\end{theorem}

\begin{proof}
\(2^kM_{k+1} : Even \Rightarrow \lnot k = 0 \vee M_{k+1} : Even\).

補題10より, \(M_{k+1} : Prime \Rightarrow k \neq 0\). よって,示された.
\end{proof}


\begin{lemma}\label{eq_two_pow_mul_odd}\lean{eq_two_pow_mul_odd}\leanok~\

任意の自然数 \(n > 0\) は,自然数 \(k\) と奇数 \(m\) を用いて, \(n = 2^k \cdot m\) と一意に表せる.

\end{lemma}

\begin{proof}
\(2 \neq 1\) かつ \(n > 0\) より, \(n\) の中に \(2\) は有限個 (\(= k\) 個) 存在する.

\(m = \frac{n}{2^k}\) とすると, \(n = 2^km\) と表せる.

今, \(\lnot 2^{k+1} \mid n\) より, \(m\) が奇数であることと対偶をとると,
\(m\) が偶数 \(\Rightarrow\) \(2^{k+1} \mid n\) となる.

\(m = 2a\) とすれば, \(n = 2^k \cdot 2a = 2^{k+1}a\) となり, \(2^{k+1} \mid n\) が成り立つ.
\end{proof}

\begin{lemma}\label{sum_properDivisors_dvd}\lean{sum_properDivisors_dvd}\leanok

自然数 \(n\) について, \(n\) の真の約数の和が \(n\) を割り切るならば, その和は \(1\) または \(n\).すなわち,
\[
\sum_{\substack{d \mid n\\ d < n}} d \mid n \quad \Rightarrow \quad \sum_{\substack{d \mid n\\ d < n}} d = 1 \vee \sum_{\substack{d \mid n\\ d < n}} d = n
\]

\end{lemma}

\begin{proof}~
\(\sum_{\substack{d \mid n\\ d < n}} d = s(n)\) と表す. \(s(n) \mid n\) と仮定する.

\(s(n) = n\) のときは明らかより, \(s(n) \neq n\) とする. このとき, \(s(n) < n\).

\(s(n) \mid n ~\text{かつ} ~ s(n) < n\) より, \(s(n)\) は \(n\) の真の約数.

今, \(s(n)\) は \(n\) の真の約数の総和であり, \(s(n)\) 自身も \(n\) の真の約数ということは,
\(n\) の真の約数は \(s(n)\) ただ1つである.
しかし, \(2\) 以上のどんな自然数に対しても, \(1\) は真の約数であるため,
真の約数が1つしか存在しないならば,それは1に限られる.
したがって \(s(n) = 1\) が成り立つ.
\end{proof}


\begin{theorem}[Euler]\label{eq_two_pow_mul_prime_mersenne_of_even_perfect}
\lean{eq_two_pow_mul_prime_mersenne_of_even_perfect}\leanok\uses{Mersenne, eq_two_pow_mul_odd,
sum_properDivisors_dvd, perfect_iff_sum_divisors_eq_two_mul, sigma_two_pow_eq_mersenne_succ, isMultiplicative}~\

\(n\) が偶数かつ完全数であるならば,ある自然数 \(k\) が存在して,
\[
n = 2^k (2^{k+1} - 1) \quad \text{かつ} \quad 2^{k+1} - 1 \text{ は素数}
\]
が成り立つ.

\end{theorem}

\begin{proof}
補題12より, \(n = 2^km\) と表せる. \(m\) は奇数より, \(2^k\) と互いに素.
よって,
\begin{align*}
  \sigma(n) &= \sigma(2^km)\\
            &= \sigma(2^k) \cdot \sigma(m)\\
            &= M_{k+1} \cdot \sigma(m)~~~(\because~lemma2.2)\\
            &= 2^{k+1}m~~~~~~~~~~(\because~n : Perfect)
\end{align*}
\(M_{k+1}\) と \(2^{k+1}\) は互いに素より,ある奇数 \(j\) を用いて, \(m = M_{k+1}j\) と表せる.

\(M_{k+1} \sigma(m) = 2^{k+1} M_{k+1}j\) より, \(\sigma(M_{k+1}j) = 2^{k+1}j\) が成り立つ. (\(\because~M_{k+1} \neq 0\))
\begin{align*}
  \sigma(M_{k+1}j) = 2^{k+1}j &\Leftrightarrow \sum_{d \mid M_{k+1}j} d = 2^{k+1}j\\
                              &\Leftrightarrow \sum_{\substack{d \mid M_{k+1}j\\ d < M_{k+1}j}} d + M_{k+1}j = 2^{k+1}j\\
                              &\Leftrightarrow \sum_{\substack{d \mid M_{k+1}j\\ d < M_{k+1}j}} d + M_{k+1}j = (M_{k+1}+1)j\\
                              &\Leftrightarrow \sum_{\substack{d \mid M_{k+1}j\\ d < M_{k+1}j}} d + M_{k+1}j = M_{k+1}j + j
\end{align*}
よって, \[\sum_{\substack{d \mid M_{k+1}j\\ d < M_{k+1}j}} d = j\] が成り立つ.

補題13より, \(\sum_{\substack{d \mid M_{k+1}j\\ d < M_{k+1}j}} d = j \mid M_{k+1}j\) となるとき,
\(\sum_{\substack{d \mid M_{k+1}j\\ d < M_{k+1}j}} d = 1~\vee~M_{k+1}j\).

\(j = 1\) のとき,
\[
  \sum_{\substack{d \mid M_{k+1}\\ d < M_{k+1}}} d = 1 \Leftrightarrow M_{k+1} : Prime
\]となり, \(M_{k+1} : Prime ~\text{かつ} ~n = 2^kM_{k+1}\) より示された.

\(j = M_{k+1}j\) のとき,
\begin{align*}
  j = M_{k+1}j &\Leftrightarrow M_{k+1} = 1\\
               &\Leftrightarrow k = 0\\
               &\text{このとき, \(n\) は奇数となるため,不適.}
\end{align*}
\end{proof}


\begin{theorem}[Euclid-Euler]\label{even_and_perfect_iff}
\lean{even_and_perfect_iff}\leanok\uses{eq_two_pow_mul_prime_mersenne_of_even_perfect,
even_two_pow_mul_mersenne_of_prime, perfect_two_pow_mul_mersenne_of_prime}~\

\( n \) を自然数とする.
メルセンヌ数 \( 2^n - 1 \) が素数であるとき, \( 2^{n-1}(2^n - 1) \) は完全数である.

逆に,任意の偶数の完全数は \( 2^{n-1}(2^n - 1) \) の形で表され,このとき \( 2^n - 1 \) はメルセンヌ素数である.

\end{theorem}

\begin{proof}
定理2.1と定理2.2より, \(M_{k+1}\) が素数ならば, \(2^k \cdot M_{k+1}\) は偶数の完全数.

定理2.3より, \(N\) が偶数かつ完全数であるならば, ある自然数 \(n\) が存在して,

\(N = 2^n(2^{n+1} - 1) \quad \text{かつ} \quad 2^{n+1} - 1 \text{は素数}\).
\end{proof}
