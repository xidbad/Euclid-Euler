% In this file you should put the actual content of the blueprint.
% It will be used both by the web and the print version.
% It should *not* include the \begin{document}
%
% If you want to split the blueprint content into several files then
% the current file can be a simple sequence of \input. Otherwise It
% can start with a \section or \chapter for instance.

\begin{definition}[完全数]\label{perfect}\lean{Perfect_}\leanok~\

自然数 \(n\) が完全数であるとは, \(n\) の真の約数(\(n\) 自身を除く正の約数)の和が \(n\) に等しいとき,すなわち
\[
\sum_{\substack{d \mid n\\ d < n}} d = n
\]
\;\;\;が成り立つことをいう.

\end{definition}


\begin{definition}[約数関数]\label{Arithmetic_Function}\lean{sigma_div}\leanok~\

自然数 \(n\) について, \(n\) のすべての約数 \(d\) に対して \(d^k\) (\(k\) は自然数) を足し合わせた関数
\[
\sigma_k(n) := \sum_{d \mid n} d^k
\]
\;\;\;を \(k\) 次の約数関数という.特に, \(\sigma_1(n)\) は \(n\) の約数の総和を表す.

\end{definition}

\vspace{0.5\baselineskip}

今回は \(k = 1\) のときのみ考え, \(\sigma_1(n) = \sigma(n)\) と表す.

\begin{lemma}

\(\sigma(n)\) は以下の性質をもつ.

\vspace{0.5\baselineskip}

(1) \(n = 1\) のときに限り, \(\sigma(n) = 1\).

(2) \(n\) が完全数のときに限り, \(\sigma(n) = 2n\).
% \label{perfect_iff_sum_divisors_eq_two_mul}
% \lean{perfect_iff_sum_divisors_eq_two_mul}\leanok

(3) \(n\) が素数のときに限り, \(\sigma(n) =  1 + n\).
% \label{prime_iff_sum_divisors_eq_succ}
% \lean[prime_iff_sum_divisors_eq_succ]\leanok

(4) 乗法的関数である.すなわち, 互いに素な自然数 \(m, n\) に対して, \(\sigma(mn) = \sigma(m) \cdot \sigma(n)\) が成り立つ.
% \label{isMultiplicative}
% \lean{isMultiplicative_sigma}\leanok

\end{lemma}

\begin{proof}

\end{proof}


\begin{definition}[メルセンヌ数]\label{mersenne}\lean{mersenne_}\leanok~\

自然数 \(k\) に対して, \(M_k := 2^k - 1\) をメルセンヌ数と呼ぶ. \(M_k\) が素数であるとき,メルセンヌ素数という.

\end{definition}


\begin{lemma}\label{sigma_two_pow_eq_mersenne_succ}\lean{sigma_two_pow_eq_mersenne_succ}\leanok

\(\sigma(2^k) = M_{k+1}\)

\end{lemma}

\begin{proof}

\end{proof}


\begin{theorem}[Euclid I]\label{perfect_two_pow_mul_mersenne_of_prime}
  \lean{perfect_two_pow_mul_mersenne_of_prime}\leanok~\

\(M_{k+1} = 2^{k+1} - 1\) が素数ならば, \(2^k \cdot M_{k+1}\) は完全数.

\end{theorem}

\begin{proof}

\end{proof}


\begin{lemma}\label{ne_zero_of_prime_mersenne}\lean{ne_zero_of_prime_mersenne}\leanok~\

\(M_{k+1} = 2^{k+1} - 1\) が素数ならば, \(k \neq 0\).

\end{lemma}

\begin{proof}

\end{proof}


\begin{theorem}[Euclid I\hspace{-1.2pt}I]\label{even_two_pow_mul_mersenne_of_prime}
\lean{even_two_pow_mul_mersenne_of_prime}\leanok~\

\(M_{k+1} = 2^{k+1} - 1\) が素数ならば, \(2^k \cdot M_{k+1}\) は偶数.

\end{theorem}

\begin{proof}

\end{proof}


\begin{lemma}\label{eq_two_pow_mul_odd}\lean{eq_two_pow_mul_odd}\leanok~\

任意の自然数 \(n > 0\) は,自然数 \(k\) と奇数 \(m\) を用いて, \(n = 2^k \cdot m\) と一意に表せる.

\end{lemma}

\begin{proof}

\end{proof}


\begin{theorem}[Euler]\label{eq_two_pow_mul_prime_mersenne_of_even_perfect}
\lean{eq_two_pow_mul_prime_mersenne_of_even_perfect}\leanok~\

\(n\) が偶数かつ完全数であるならば,ある自然数 \(k\) が存在して,
\[
n = 2^k (2^{k+1} - 1)
\quad \text{かつ} \quad
2^{k+1} - 1 \text{ は素数}
\]
が成り立つ.

\end{theorem}

\begin{proof}

\end{proof}


\begin{theorem}[Euclid-Euler]\label{even_and_perfect_iff}
\lean{even_and_perfect_iff}\leanok~\

\( n \) を自然数とする.
メルセンヌ数 \( 2^n - 1 \) が素数であるとき, \( 2^{n-1}(2^n - 1) \) は完全数である.

逆に,任意の偶数の完全数は \( 2^{n-1}(2^n - 1) \) の形で表され,このとき \( 2^n - 1 \) は\(メルセンヌ\)素数である.

\end{theorem}

\begin{proof}

\end{proof}
