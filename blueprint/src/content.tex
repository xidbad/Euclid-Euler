% In this file you should put the actual content of the blueprint.
% It will be used both by the web and the print version.
% It should *not* include the \begin{document}
%
% If you want to split the blueprint content into several files then
% the current file can be a simple sequence of \input. Otherwise It
% can start with a \section or \chapter for instance.

\begin{definition}[完全数]\label{perfect}\lean{Perfect_}\leanok~\

自然数 \(n\) が完全数であるとは, \(n\) の真の約数(\(n\) 自身を除く正の約数)の総和が \(n\) に等しいとき,すなわち
\[ \sum_{\substack{d \mid n\\ d < n}} d = n \]
が成り立つことをいう.

\end{definition}


\begin{definition}[約数関数]\label{sigma_div}\lean{sigma_div}\leanok~\

自然数 \(n\) について, \(n\) のすべての約数 \(d\) に対して \(d^k\) (\(k\) は自然数) を足し合わせた関数
\[ \sigma_k(n) := \sum_{d \mid n} d^k \]
を \(k\) 次の約数関数という.

\end{definition}


\begin{lemma}\label{sigma_one_apply}\lean{sigma_one_apply_}\leanok\uses{sigma_div}
\(k = 1\) のとき, \(\sigma_k(n)\) は \(n\) の約数の総和を表す.
\end{lemma}

\begin{proof}
約数関数の定義から明らか.
\end{proof}


今回は \(k = 1\) のときのみ考え, \(\sigma(n) = \sigma(n)\) と表す.
\(\sigma(n)\) は以下の性質をもつ.


\begin{lemma}\label{one_iff_sum_divisors_eq_one}\lean{one_iff_sum_divisors_eq_one}
\leanok\uses{sigma_one_apply}
\(n = 1\) のときに限り, \(\sigma(n) = 1\).
\end{lemma}

\begin{proof}
\(n = 1\) のとき, \(n\) の約数は \{1\}のみ. \(\therefore ~\sigma(n) = 1\).

逆に, \(\sigma(n) = 1\) のとき, \(n \neq 1\) とすると, \(\sigma(n) > 1\) となるため, \(n = 1\).
\end{proof}


\begin{lemma}\label{perfect_iff_sum_divisors_eq_two_mul}
\lean{perfect_iff_sum_divisors_eq_two_mul}\leanok\uses{perfect, sigma_one_apply}
\(n\) が完全数のときに限り, \(\sigma(n) = 2n\).
\end{lemma}

\begin{proof}
\(s(n) = \sum_{\substack{d \mid n\\ d < n}} d = n\) とする.
Definition1より, \(n\) が完全数のとき, \(s(n) = n\).

\(\sigma(n) = s(n) + n\) であるから, \(\sigma(n) = 2n\).

逆に, \(\sigma(n) = s(n) + n = 2n\) より, \(s(n) = n\).
よって, \(n\) は完全数.
\end{proof}

\begin{lemma}\label{prime_iff_sum_divisors_eq_succ}\lean{prime_iff_sum_divisors_eq_succ}
\leanok\uses{sigma_one_apply}
\(n\) が素数のときに限り, \(\sigma(n) =  1 + n\).
\end{lemma}

\begin{proof}
\(n\) が素数のとき, \(n\) の約数は \{1, n\} より, \(\sigma(n) = 1 + n\).

逆に, \(\sigma(n) = s(n) + n = 1 + n\) より, \(s(n) = 1\).
このとき, \(n\) は素数.
\end{proof}

\begin{lemma}\label{isMultiplicative}\lean{isMultiplicative_sigma}\leanok\uses{sigma_one_apply}
\(\sigma(n)\) は乗法的関数である.すなわち, 互いに素な自然数 \(m, n\) に対して,
\(\sigma(mn) = \sigma(m)\sigma(n)\) が成り立つ.
\end{lemma}

\begin{proof}
まず,ゼータ関数とpow関数を次のように定義する.これらは数論的関数(Arithmetic Function)と呼ばれる.
数論的関数とは, 定義域が自然数の関数である. (Leanだと0は自然数に含まれるため, 0は0に飛ぶことにする)
\[\zeta(n) :=
\begin{cases} 1 & (n \geqq 1)\\
              0 & (n = 0)
\end{cases}\]
\[\mathrm{pow}_k(n) :=
\begin{cases} n^k & (n \geqq 1)\\
                0 & (n = 0)
\end{cases}\]

\(f, g\) を数論的関数とすると,数論的関数同士のディリクレ積は次のように定義される.
\[(f \ast g)(n) := \sum_{d \mid n} f(d)g(\frac{n}{d})\]

ここで, \(f = \zeta, ~g = \mathrm{pow}_k\) とすると,
\begin{align*}
  (\zeta \ast \mathrm{pow}_k)(n) ~&=~ \sum_{d \mid n} \zeta(d) \mathrm{pow}_k(\frac{n}{d})\\
                                 ~&=~ \sum_{d \mid n} 1 \cdot (\frac{n}{d})^k~~(\because~\zeta(d) = 1 (d \geqq 1))\\
                                 ~&=~ \sum_{d \mid n} d ^ k~~(\because ~\frac{n}{d} \text{は} n \text{の約数})\\
                                 ~&=~ \sigma_k(n)
\end{align*}

乗法的関数同士のディリクレ積もまた乗法的関数であることを示す.
\begin{proof}
\(f, g\) を乗法的関数とする.互いに素な自然数 \(m, n\) に対して,
\(mn\) の約数 \(d\) は \(d = d_1d_2 ~(d_1 \mid m, d_2 \mid n)\) と一意に表せる.
\begin{align*}
  (f \ast g)(mn) ~&=~ \sum_{d \mid mn} f(d)g(\frac{mn}{d})\\
                 ~&=~ \sum_{d_1d_2 \mid mn} f(d_1d_2)g(\frac{mn}{d_1d_2})\\
                 ~&=~ \sum_{d_1 \mid m} \sum_{d_2 \mid n} f(d_1)f(d_2)g(\frac{m}{d_1})g(\frac{n}{d_2})~~(\because ~d_1, d_2 \text{は互いに素})\\
                 ~&=~ \{\sum_{d_1 \mid m} f(d_1)g(\frac{m}{d_1})\} \cdot \{\sum_{d_2 \mid n} f(d_2)g(\frac{n}{d_2})\}\\
                 ~&=~ (f \ast g)(m) \cdot (f \ast g)(n)
\end{align*}\end{proof}
\(\zeta\) 関数も \(\mathrm{pow}_k\) 関数も乗法的関数であり,それらのディリクレ積もまた乗法的関数であるため,
\(\sigma_k\) 関数も乗法的関数である.
\end{proof}


\begin{definition}[メルセンヌ数]\label{Mersenne}\lean{Mersenne}\leanok~\

自然数 \(k\) に対して, \(M_k := 2^k - 1\) をメルセンヌ数と呼ぶ. \(M_k\) が素数であるとき,メルセンヌ素数という.

\end{definition}


\begin{lemma}\label{sigma_two_pow_eq_mersenne_succ}\lean{sigma_two_pow_eq_mersenne_succ}\leanok
\uses{Mersenne}
\(\sigma(2^k) = M_{k+1}\)
\end{lemma}

\begin{proof}
\begin{align*}
  \sigma(2^k) = M_{k+1} ~&\Leftrightarrow~ \sum_{d \mid 2^k} d = M_{k+1}\\
                        ~&\Leftrightarrow~ \sum_{d \mid 2^k} d = 2^{k+1} - 1\\
                        ~&\Leftrightarrow~ \sum_{d \mid (1+1)^k} d = (1+1)^{k+1} - 1\\
                        ~&\Leftrightarrow~ \sum_{d \mid (1+1)^k} d = \{(\sum_{i=0}^{k}(1+1)^i) \cdot 1 + 1\} - 1
                        ~(\because ~\sum_{i=0}^{n} (x+1)^i = \frac{(x+1)^{n+1}-1}{(x+1)-1})\\
                        ~&\Leftrightarrow~ \sum_{d \mid (1+1)^k} d = \sum_{i=0}^{k} (1+1)^i
\end{align*}
\end{proof}


\begin{theorem}[Euclid I]\label{perfect_two_pow_mul_mersenne_of_prime}
\lean{perfect_two_pow_mul_mersenne_of_prime}\leanok
\uses{perfect_iff_sum_divisors_eq_two_mul, sigma_two_pow_eq_mersenne_succ, Mersenne, prime_iff_sum_divisors_eq_succ, isMultiplicative}~\

\(M_{k+1} = 2^{k+1} - 1\) が素数ならば, \(2^kM_{k+1}\) は完全数.

\end{theorem}

\begin{proof}
\begin{align*}
2^kM_{k+1} : Perfect ~&\Leftrightarrow~ \sum_{d \mid 2^kM_{k+1}} d = 2(2^kM_{k+1}) ~\wedge~ 2^kM_{k+1} > 0\\
                     ~&\Leftrightarrow~ \sum_{d \mid 2^kM_{k+1}} d = 2^{k+1}M_{k+1} ~\wedge~ 2^kM_{k+1} > 0\\
                     ~&\Leftrightarrow~ \sigma(2^kM_{k+1}) = 2^{k+1}M_{k+1} ~\wedge~ 2^kM_{k+1} > 0\\
                     ~&\Leftrightarrow~ \sigma(M_{k+1})\sigma(2^k) = 2^{k+1}M_{k+1} ~\wedge~ 2^kM_{k+1} > 0~~(\because~Lemma7)\\
                     ~&\Leftrightarrow~ \sigma(M_{k+1})M_{k+1} = 2^{k+1}M_{k+1} ~\wedge~ 2^kM_{k+1} > 0~~(\because~Lemma9)\\
\end{align*}
\(M_{k+1}\) は素数より, Lemma6から \(\sigma(M_{k+1}) = 1 + M_{k+1} = 2^{k+1}\) かつ \(2^kM_{k+1} > 0\).
\end{proof}


\begin{lemma}\label{ne_zero_of_prime_mersenne}\lean{ne_zero_of_prime_mersenne}\leanok
\uses{Mersenne}
\(M_{k+1} = 2^{k+1} - 1\) が素数ならば, \(k \neq 0\).
\end{lemma}

\begin{proof}
背理法を用いる. \(k = 0\) と仮定する.

\(M_{k+1} : Prime = M_{0+1} = 1\) となるが, \(1\) は素数でない.よって矛盾.
\end{proof}


\begin{theorem}[Euclid I\hspace{-1.2pt}I]\label{even_two_pow_mul_mersenne_of_prime}
\lean{even_two_pow_mul_mersenne_of_prime}\leanok\uses{ne_zero_of_prime_mersenne, Mersenne}~\

\(M_{k+1} = 2^{k+1} - 1\) が素数ならば, \(2^kM_{k+1}\) は偶数.

\end{theorem}

\begin{proof}
\(2^kM_{k+1} : Even \Leftrightarrow \lnot k = 0 ~\vee ~M_{k+1} : Even\).

Lemma11より, \(M_{k+1} : Prime \Rightarrow k \neq 0\).
\end{proof}


\begin{lemma}\label{eq_two_pow_mul_odd}\lean{eq_two_pow_mul_odd}\leanok
任意の自然数 \(n > 0\) は,ある自然数 \(k, m\) を用いて, \(n = 2^km\) と表せ,このとき \(m\) は奇数.
\end{lemma}

\begin{proof}
\(2 \neq 1\) かつ \(n > 0\) より, \(n\) の中に \(2\) は有限個存在する.

その個数を \(k\) 個とすると, \(m = \frac{n}{2^k}\) とすることで, \(n = 2^km\) と表せる.

次に \(m\) が奇数であることを示す.

今, \(\lnot 2^{k+1} \mid n\) より, \(m\) が奇数であることと対偶をとると,
\(m\) が偶数 \(\Rightarrow\) \(2^{k+1} \mid n\) となる.

\(m = 2a\) とすれば, \(n = 2^k2a = 2^{k+1}a\) となり, \(2^{k+1} \mid n\) が成り立つ.
\end{proof}

\begin{lemma}\label{sum_properDivisors_dvd}\lean{sum_properDivisors_dvd}\leanok

自然数 \(n\) について, \(n\) の真の約数の和が \(n\) を割り切るならば, その和は \(1\) または \(n\).すなわち,
\[
\sum_{\substack{d \mid n\\ d < n}} d \mid n \quad \Rightarrow \quad \sum_{\substack{d \mid n\\ d < n}} d = 1 ~\vee~ \sum_{\substack{d \mid n\\ d < n}} d = n
\]
が成り立つ.
\end{lemma}

\begin{proof}~
\(\sum_{\substack{d \mid n\\ d < n}} d = s(n)\) と表す. \(s(n) \mid n\) と仮定する.

\(s(n) = n\) のときは明らかより, \(s(n) \neq n\) とする. このとき, 仮定から \(s(n) < n\).

\(s(n) \mid n ~\text{かつ} ~ s(n) < n\) より, \(s(n)\) は \(n\) の真の約数.

今, \(s(n)\) は \(n\) の真の約数の総和であり, \(s(n)\) 自身も \(n\) の真の約数ということは,
\(n\) の真の約数は \(s(n)\) ただ1つである.
しかし, \(2\) 以上のどんな自然数に対しても, \(1\) は真の約数であるため,
真の約数が1つしか存在しないならば,それは1に限られる.
したがって \(s(n) = 1\) が成り立つ.
\end{proof}


\begin{theorem}[Euler]\label{eq_two_pow_mul_prime_mersenne_of_even_perfect}
\lean{eq_two_pow_mul_prime_mersenne_of_even_perfect}\leanok\uses{Mersenne, eq_two_pow_mul_odd,
sum_properDivisors_dvd, perfect_iff_sum_divisors_eq_two_mul, sigma_two_pow_eq_mersenne_succ, isMultiplicative}~\

\(n\) が偶数かつ完全数ならば,ある自然数 \(k\) が存在して,
\[
n = 2^k (2^{k+1} - 1) ~\vee~ 2^{k+1} - 1 \text{は素数}
\]
が成り立つ.

\end{theorem}

\begin{proof}
Lemma13より, \(n = 2^km\) と表せる. \(m\) は奇数より, \(2^k\) と互いに素.
よって,
\begin{align*}
\sigma(n) ~&=~ \sigma(2^km)\\
          ~&=~ \sigma(2^k)\sigma(m)~~(\because~Lemma7)\\
          ~&=~ M_{k+1}\sigma(m)~~(\because~Lemma9)
\end{align*}

また \(n\) は完全数より, Lemma5から \(\sigma(n) = 2^{k+1}m\). \(\therefore~ M_{k+1}\sigma(m) = 2^{k+1}m\).

\(M_{k+1}\) と \(2^{k+1}\) は互いに素より,ある奇数 \(j\) を用いて, \(m = M_{k+1}j\) と表せる.

\(M_{k+1} \sigma(m) = 2^{k+1} M_{k+1}j\) より, \(\sigma(M_{k+1}j) = 2^{k+1}j\) が成り立つ. (\(\because~M_{k+1} \neq 0\))
\begin{align*}
\sigma(M_{k+1}j) = 2^{k+1}j ~&\Leftrightarrow~ \sum_{d \mid M_{k+1}j} d = 2^{k+1}j\\
                            ~&\Leftrightarrow~ \sum_{\substack{d \mid M_{k+1}j\\ d < M_{k+1}j}} d + M_{k+1}j = 2^{k+1}j\\
                            ~&\Leftrightarrow~ \sum_{\substack{d \mid M_{k+1}j\\ d < M_{k+1}j}} d + M_{k+1}j = (M_{k+1}+1)j\\
                            ~&\Leftrightarrow~ \sum_{\substack{d \mid M_{k+1}j\\ d < M_{k+1}j}} d + M_{k+1}j = M_{k+1}j + j\\
                            ~&\Leftrightarrow~ \sum_{\substack{d \mid M_{k+1}j\\ d < M_{k+1}j}} d = j
\end{align*}

よって, \(\sum_{\substack{d \mid M_{k+1}j\\ d < M_{k+1}j}} d \mid M_{k+1}j\) であるから, Lemma14より, \(j = 1 ~\vee~ j = M_{k+1}j\).

\(j = 1\) のとき,
\[
\sum_{\substack{d \mid M_{k+1}\\ d < M_{k+1}}} d = 1 \Leftrightarrow M_{k+1} : Prime~~(\because~Lemma6)
\]
となり, \(M_{k+1} : Prime ~\text{かつ} ~n = 2^kM_{k+1}\) より示された.

\(j = M_{k+1}j\) のとき,
\begin{align*}
j = M_{k+1}j ~&\Leftrightarrow~ M_{k+1} = 1\\
             ~&\Leftrightarrow~ k = 0
\end{align*}
このとき \(n = 2^km = m\) で, \(n\) が奇数となるため,不適.
\end{proof}


\begin{theorem}[Euclid-Euler]\label{even_and_perfect_iff}
\lean{even_and_perfect_iff}\leanok\uses{eq_two_pow_mul_prime_mersenne_of_even_perfect,
even_two_pow_mul_mersenne_of_prime, perfect_two_pow_mul_mersenne_of_prime}~\

\(k\) を自然数とする.

メルセンヌ数 \(2^{k+1} - 1\) が素数であるとき, \(2^k(2^{k+1} - 1)\) は偶数の完全数である.

逆に,任意の偶数の完全数は \(2^k(2^{k+1} - 1)\) の形で表され,このとき \(2^{k+1} - 1\) はメルセンヌ素数である.

\end{theorem}

\begin{proof}
Theorem10とTheorem12より, \(M_{k+1}\) が素数ならば, \(2^kM_{k+1}\) は偶数の完全数.

Theorem15より, \(N\) が偶数かつ完全数であるならば, ある自然数 \(k\) が存在して,

\(N = 2^k(2^{k+1} - 1)\) ~かつ~ \(2^{k+1} - 1 \text{は素数}\).
\end{proof}
